\hypertarget{group___a_d_c___private___macros}{\section{A\-D\-C\-\_\-\-Private\-\_\-\-Macros}
\label{group___a_d_c___private___macros}\index{A\-D\-C\-\_\-\-Private\-\_\-\-Macros@{A\-D\-C\-\_\-\-Private\-\_\-\-Macros}}
}
