\hypertarget{group___a_d_c__data__align}{\section{A\-D\-C\-\_\-data\-\_\-align}
\label{group___a_d_c__data__align}\index{A\-D\-C\-\_\-data\-\_\-align@{A\-D\-C\-\_\-data\-\_\-align}}
}
