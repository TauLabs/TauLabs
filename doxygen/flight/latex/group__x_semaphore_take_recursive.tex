\hypertarget{group__x_semaphore_take_recursive}{\section{x\-Semaphore\-Take\-Recursive}
\label{group__x_semaphore_take_recursive}\index{x\-Semaphore\-Take\-Recursive@{x\-Semaphore\-Take\-Recursive}}
}
semphr. h x\-Semaphore\-Take\-Recursive( x\-Semaphore\-Handle x\-Mutex, port\-Tick\-Type x\-Block\-Time )

{\itshape Macro} to recursively obtain, or 'take', a mutex type semaphore. The mutex must have previously been created using a call to x\-Semaphore\-Create\-Recursive\-Mutex();

config\-U\-S\-E\-\_\-\-R\-E\-C\-U\-R\-S\-I\-V\-E\-\_\-\-M\-U\-T\-E\-X\-E\-S must be set to 1 in Free\-R\-T\-O\-S\-Config.\-h for this macro to be available.

This macro must not be used on mutexes created using x\-Semaphore\-Create\-Mutex().

A mutex used recursively can be 'taken' repeatedly by the owner. The mutex doesn't become available again until the owner has called x\-Semaphore\-Give\-Recursive() for each successful 'take' request. For example, if a task successfully 'takes' the same mutex 5 times then the mutex will not be available to any other task until it has also 'given' the mutex back exactly five times.


\begin{DoxyParams}{Parameters}
{\em x\-Mutex} & A handle to the mutex being obtained. This is the handle returned by x\-Semaphore\-Create\-Recursive\-Mutex();\\
\hline
{\em x\-Block\-Time} & The time in ticks to wait for the semaphore to become available. The macro port\-T\-I\-C\-K\-\_\-\-R\-A\-T\-E\-\_\-\-M\-S can be used to convert this to a real time. A block time of zero can be used to poll the semaphore. If the task already owns the semaphore then x\-Semaphore\-Take\-Recursive() will return immediately no matter what the value of x\-Block\-Time.\\
\hline
\end{DoxyParams}
\begin{DoxyReturn}{Returns}
pd\-T\-R\-U\-E if the semaphore was obtained. pd\-F\-A\-L\-S\-E if x\-Block\-Time expired without the semaphore becoming available.
\end{DoxyReturn}
Example usage\-: 
\begin{DoxyPre}
 xSemaphoreHandle xMutex = NULL;\end{DoxyPre}



\begin{DoxyPre}A task that creates a mutex.
 void vATask( void * pvParameters )
 \{
Create the mutex to guard a shared resource.
    xMutex = xSemaphoreCreateRecursiveMutex();
 \}\end{DoxyPre}



\begin{DoxyPre}A task that uses the mutex.
 void vAnotherTask( void * pvParameters )
 \{
... Do other things.
\begin{DoxyVerb}if( xMutex != NULL )
{
\end{DoxyVerb}

See if we can obtain the mutex.  If the mutex is not available
wait 10 ticks to see if it becomes free.        
        if( xSemaphoreTakeRecursive( xSemaphore, ( portTickType ) 10 ) == pdTRUE )
        \{
We were able to obtain the mutex and can now access the
shared resource.\end{DoxyPre}



\begin{DoxyPre}...
For some reason due to the nature of the code further calls to 
xSemaphoreTakeRecursive() are made on the same mutex.  In real
code these would not be just sequential calls as this would make
no sense.  Instead the calls are likely to be buried inside
a more complex call structure.
            xSemaphoreTakeRecursive( xMutex, ( portTickType ) 10 );
            xSemaphoreTakeRecursive( xMutex, ( portTickType ) 10 );\end{DoxyPre}



\begin{DoxyPre}The mutex has now been 'taken' three times, so will not be 
available to another task until it has also been given back
three times.  Again it is unlikely that real code would have
these calls sequentially, but instead buried in a more complex
call structure.  This is just for illustrative purposes.
            xSemaphoreGiveRecursive( xMutex );
                        xSemaphoreGiveRecursive( xMutex );
                        xSemaphoreGiveRecursive( xMutex );\end{DoxyPre}



\begin{DoxyPre}Now the mutex can be taken by other tasks.
        \}
        else
        \{
We could not obtain the mutex and can therefore not access
the shared resource safely.
        \}
    \}
 \}
 \end{DoxyPre}
 