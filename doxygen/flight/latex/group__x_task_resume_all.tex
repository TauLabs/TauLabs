\hypertarget{group__x_task_resume_all}{\section{x\-Task\-Resume\-All}
\label{group__x_task_resume_all}\index{x\-Task\-Resume\-All@{x\-Task\-Resume\-All}}
}
task. h 
\begin{DoxyPre}char \hyperlink{win32_2win32_2_libraries_2_free_r_t_o_s_2_source_2tasks_8c_ab173f313b22f1a57dfceb98cb01b0508}{xTaskResumeAll( void )};\end{DoxyPre}


Resumes real time kernel activity following a call to v\-Task\-Suspend\-All (). After a call to v\-Task\-Suspend\-All () the kernel will take control of which task is executing at any time.

\begin{DoxyReturn}{Returns}
If resuming the scheduler caused a context switch then pd\-T\-R\-U\-E is returned, otherwise pd\-F\-A\-L\-S\-E is returned.
\end{DoxyReturn}
Example usage\-: 
\begin{DoxyPre}
 void vTask1( void * pvParameters )
 \{
         for( ;; )
         \{
Task code goes here.\end{DoxyPre}



\begin{DoxyPre}...\end{DoxyPre}



\begin{DoxyPre}At some point the task wants to perform a long operation during
which it does not want to get swapped out.  It cannot use
taskENTER\_CRITICAL ()/taskEXIT\_CRITICAL () as the length of the
operation may cause interrupts to be missed - including the
ticks.\end{DoxyPre}



\begin{DoxyPre}Prevent the real time kernel swapping out the task.
                 vTaskSuspendAll ();\end{DoxyPre}



\begin{DoxyPre}Perform the operation here.  There is no need to use critical
sections as we have all the microcontroller processing time.
During this time interrupts will still operate and the real
time kernel tick count will be maintained.\end{DoxyPre}



\begin{DoxyPre}...\end{DoxyPre}



\begin{DoxyPre}The operation is complete.  Restart the kernel.  We want to force
a context switch - but there is no point if resuming the scheduler
caused a context switch already.
                 if( !xTaskResumeAll () )
                 \{
                          taskYIELD ();
                 \}
         \}
 \}
   \end{DoxyPre}
 